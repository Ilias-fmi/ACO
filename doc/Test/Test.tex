\chapter{Tests}
\minitoc

Hier werden unsere Testfälle tabellarisch aufgelistet mit denen wir unser Plugin unter Ilias 5.2.2 getestet haben. \\
Die Tests sind nach den Tabs uns Subtabs aufgeteilt. Voraussetzung um diese Test durchführen zu können ist die erfolgreiche Installation und Aktivierung des Plugins.  
Der Haken bedeutet, dass das Istresultat dem Sollresultat entsprochen hat. \\

\begin{table}[]
	\centering
	\caption{Testfälle}
	\label{table1}
	\begin{tabular}{p{6cm}p{7cm}p{3cm}llll}
	\textbf{\large{Testfälle Plugin ACO } }                                                                                                              &                                                                                                          &             &  &  &  \\
		&                                                                                                                                                                  &             &  &  &  \\
		\textbf{Tab Kurs verwalten}                                                                                                                   &                                                                                                                                                                  &             &  &  &  \\
		\textbf{Subtab Gruppen erstellen }                                                                                                            &                                                                                                                                                                  &             &  &  &  \\
		&                                                                                                                                                                  &             &  &  &  \\
		\textbf{Testfall  }                                                                                                                           & \textbf{Sollresultat  }                                                                                                                                                   &\textbf{ Istresultat} &  &  &  \\
		Klicke auf Kurs verwalten                                                                                                            & Weiterführung auf neue Seite mit den Subtabs Gruppen erstellen/ verwalten/ Mitglieder verschieben                                                                & \checkmark           &  &  &  \\
		Klicken auf den Knopf zurück                                                                                                         & Führt einen zurück zum Kurs in dem man den Tab aufgerufen hat                                                                                                    & 
		\checkmark          &  &  &  \\
		Erstelle in neuen Kurs 12 Gruppen (Name: Gruppe) mit max. Mitglieder 15 und direktem Beitritt                                        & 12 Gruppen "Gruppe" fortlaufend nummeriert mit max. Mitgliedern 15 und direktem Beitritt                                                                         & \checkmark           &  &  &  \\
		Erstelle in Kurs in dem schon 12 Gruppen existieren 5 neue mit den selben Einstellungen wie oben                                     & Erstellt 5 neue Gruppen mit fortlaufender Nummer, max. Mitglieder 15 und direktem Beitritt                                                                        & \checkmark           &  &  &  \\
		Erstelle eine Gruppe mit zeitlich begrenzten Beitritt Start: 27.03.17 15:30 Ende: 09.12.17 00:01                                     & Erstellt diese Gruppe mit fortlaufender Nummer und diesem Regstrierungszeitraum                                                                                   & \checkmark           &  &  &  \\
		Erstelle zwei Gruppen mit Passwortgeschütztem Beitritt, pw: "passwort"                                                               & Erstellt zwei Gruppen mit diesem Passwort                                                                                                                        & \checkmark          &  &  &  \\
		Erstelle eine neue Gruppe mit Ordnerstruktur: Admin Folder                                                                           & Erstellt Gruppe mit fortlaufender Nummer und AdminFolder im Kurs und in jeder existierenden Gruppe                                                               & \checkmark           &  &  &  \\
      	Lösche eine Gruppe im Kurs und erstelle danach im Plugin eine neue                                                                   & Neue Gruppe wird mit derselben Nummer erstellt wie die gelöschte                                                                                                 & \checkmark           &  &  &  \\
		Erstelle Gruppe mit Buchstaben in Gruppenanzahl                                                                                      & Fehlermeldung: Erstellen der Gruppen fehlgeschlagen, überprüfen Sie ihre Eingaben.                                                                               & \checkmark           &  &  &  \\
		Erstelle Gruppe mit Zahl gefolgt von Buchstaben im Max Mitglieder Feld                                                               & Ignoriere Buchstaben hinter der Zahl und erstelle Gruppe mit dieser Anzahl max. Mitglieder                                                                         & \checkmark           &  &  &  \\
		Erstelle Gruppe mit Buchstaben in Anzahl Gruppen                                                                                     & Fehlermeldung "Erstellen der Gruppen fehlgeschlagen, überprüfen Sie ihre Eingaben."  wird angezeigt und keine Gruppe erstellt                                    & \checkmark           &  &  &  \\
		Erstelle Gruppe mit Buchstaben hinter ersten Zahl in Anzahl Gruppen                                                                  & System ignoriert Eingabe hinter dem ersten Buchstaben                                                                                                           & \checkmark           &  &  &  \\
		Erstelle Gruppe mit Buchstaben in max Mitglieder                                                                                     & Interpretiere Bustaben als 0 und erzeuge Gruppe ohne Beschränkung                                                                                                & \checkmark           &  &  &  \\
		Erzeuge Gruppe mit Ordnerstruktur (Admin Folder)                                                                                     & Gruppe und der Ordner im Kurs als auch in der Gruppe wurden erzeugt                                                                                              & \checkmark           &  &  &  \\
			\end{tabular}
\end{table}

\begin{table}[]
	\centering
	\caption{Testfälle}
	\label{my-label}
	\begin{tabular}{p{6cm}p{7cm}p{3cm}llll}
		Erzeuge Gruppe mit Ordnerstruktur (AdminFolder und Blätter)  obwohl Admin Folder schon im Kurs existiert aber noch in keiner Gruppe  & Gruppe wurde erzeugt und in jeder Gruppe befinden sich beide Ordner 2 mal, im Kurs selbst auch, dh. es gibt kein AdminFolder Duplikat                            & \checkmark           &  &  &  \\
		Erstelle Gruppe mit Anmeldezeitraumende vor Anmeldezeitraumstart                                                                     & Gruppe wird mit Anmeldezeitraumstart = Anmeldezeitraumende erstellt                                                                                              & \checkmark           &  &  &  \\

Teste mit einem User ob dieser Zugriff auf die Plugin Tabs hat&      User hat keinen Zugriff auf die Plugin Tabs                                                                                                                                                            & \checkmark                &  &  &  \\
&                                                                                                                                                                  &             &  &  &  \\
		Subtab Kurs bearbeiten und Mitglieder verschieben aufrufen                                                                           & Weiterführung auf die entsprechenden Sub tabs                                                                                                                    & \checkmark           &  &  &  \\
		&                                                                                                                                                                  &             &  &  &  \\
		&                                                                                                                                                                  &             &  &  &  \\
		\textbf{Subtab Kurs bearbeiten     }                                                                                                          &                                                                                                                                                                  &             &  &  &  \\
		&                                                                                                                                                                  &             &  &  &  \\
	\textbf{	Testfall }                                                                                                                            &    \textbf{Sollresultat}                                                                                                                                                              & \textbf{Istresultat}            &  &  &  \\
		Subtab Kurs bearbeiten aufrufen                                                                                                      &                                                                                                                                                                  &             &  &  &  \\
		Klicken in den drei Subtabs auf den Knopf zurück                                                                                     & Führt einen zurück zum Kurs in dem man den Tab aufgerufen hat                                                                                                    & \checkmark           &  &  &  \\
		In Kurs mit bestehenden Gruppen nichts verändern und auf speichern drücken                                                            & Einstellungen bleiben gleich                                                                                                                                     & \checkmark           &  &  &  \\
		Gruppenname einer bestehenden Gruppe ändern inkl Nummer und Sonderzeichen                                                      & übernimmt neuen Namen für die Gruppe (unabh ob man nur Buchstaben eingibt oder gar nur Sonderzeichen)                                                            & \checkmark           &  &  &  \\
		ändere leeres Feld Raum und Uhrzeit auf Raum 0.125 8 Uhr                                                                           & übernimmt die Eingabe und zeigt sie in der Gruppenbeschreibung an                                                                                                & \checkmark           &  &  &  \\
		Ändere Max Mitglieder auf 8                                                                                                          & ändere max. Mitglieder genau dieser Gruppe auf 8                                                                                                                 & \checkmark           &  &  &  \\
		Besuche Sub Tab bei Kurs ohne Gruppen und mit genau 22 Gruppen, lösche dann eine Gruppe bei den 22 und füge bei dem ohne eine hinzu & Zeigt 22 Gruppen, keine, eine und 21 und deren Einstellungen an                                                                                                  & \checkmark           &  &  &  \\
		Eingabe von Buchstaben, Sonderzeichen, 0 oder leerer Eingabe  in max Mitglieder                                                      & Interpretiert all diese Eingaben als 0 und übernimmt diesen Wert                                                                                                 & \checkmark           &  &  &  \\
		Bearbeite Gruppe mit Buchstaben hinter ersten Zahl in Anzahl Gruppen                                                                 & System ignoriert Eingabe hinter dem ersten Buchstaben                                                                                                           & \checkmark           &  &  &  \\
		Bearbeite Gruppe mit Buchstaben in max Mitglieder                                                                                    & Interpretiere Buchstaben als 0 und erzeuge Gruppe ohne Beschränkung                                                                                                & \checkmark           &  &  &  \\
		Füge zeitlich begrenzten Beitritt bei einer Gruppe hinzu                                                                              & übernimmt den festgelegten Anmeldezeitraumstart und Ende                                                                                                         & \checkmark           &  &  &  \\
	
		
	\end{tabular}
\end{table}
\begin{table}[]
	\centering
	\caption{Testfälle}
	\label{table2}
	\begin{tabular}{p{6cm}p{7cm}p{3cm}llll}
		Bearbeite Gruppe mit Anmeldezeitraumende vor Anmeldezeitraumstart                                                                    & Gruppe wird mit Anmeldezeitraumstart und Anmeldezeitraumende + 2 Tage übernommen                                                                                 & \checkmark            &  &  &  \\
		
		Entferne Checkboxhaken bei einer Gruppe mit zeitlich begrenztem Beitritt                                                             & Gruppe hat keinen begrenzten Beitritt mehr                                                                                                                       & \checkmark           &  &  &  \\
		Trage einen Benutzer der nicht in der entsprechenden Gruppe ist unter Tutor ein                                                      & Füge Benutzer in der Gruppe hinzu und gebe ihm Adminrechte (Autocomplete hat funktioniert)                                                                       & \checkmark           &  &  &  \\
		Entferne Eintrag im Tutorfeld bzw. den Tutor                                                                                         & Fehlermeldung und übernimmt wieder alten tutor                                                                                                                   &             &  &  &  \\
		Trage einen Benutzer welcher in der entsprechenden Gruppe ist unter Tutor ein                                                        & Benutzer hat jetzt Tutorrechte bzw Gruppenadminrechte (Autocomplete hat funktioniert)                                                                            & \checkmark           &  &  &  \\
		Führe obere Tests für mehrere Gruppen die existieren gleichzeitig aus                                                                & Führt die Tests genauso aus nur für alle Gruppen die geändert wurden                                                                                             & \checkmark           &  &  &  \\

		Subtab Gruppen erstellen und Mitglieder verschieben aufrufen                                                                         & Weiterführung auf die entsprechenden Sub tabs                                                                                                                    & \checkmark           &  &  &  \\&                                                                                                                                                                  &             &  &  &  \\
	
	
			\textbf{Subtab Mitglieder verschieben}                                                                                                        &                                                                                                                                                                  &             &  &  &  \\
		&                                                                                                                                                                  &             &  &  &  \\
			\textbf{Testfall }                                                                                                                            &       	\textbf{Sollresultat}                                                                                                                                                           &      	\textbf{Istresultat}       &  &  &  \\
		Klicken in den drei Subtabs auf den Knopf zurück                                                                                     & Führt einen zurück zum Kurs in dem man den Tab aufgerufen hat                                                                                                    & \checkmark           &  &  &  \\
		Gebe Nutzernamen ein der in Gruppe 1 und 2 ist und verschiebe ihn aus Gruppe 1 in Gruppe 3                                           & Zeige in neuer Seite Benutzernamen und bisherige Gruppen sowie im drop down darunter alle ex. Gruppen an (Autocomplete hat funktioniert), Nutzer wurde verschoben & \checkmark           &  &  &  \\
		Testfall 1 und Anmeldungsbeginn in der Zukunft                                                                                       & Verschiebt Nutzer trotzdem                                                                                                                                       & \checkmark           &  &  &  \\
		Testfall 1 und Anmeldungsende schon vorbei                                                                                           & Verschiebt den Nutzer trotzdem                                                                                                                                   & \checkmark           &  &  &  \\
		Testfall 1 und max. Mitglieder überschritten (keine Plätze mehr verfügbar)                                                           & Verschiebt Nutzer trotzdem                                                                                                                                       & \checkmark           &  &  &  \\
		Nutzer noch in keiner Gruppe, soll verschoben werden                                                                                 & Dropdown bisherige Gruppe leer, Fehlermeldung "Gruppe existiert nicht"                                                                                           & \checkmark           &  &  &  \\
		Nutzer nicht im Kurs, will ihn trotzdem verschieben                                                                                  & Dropdown bisherige Gruppe leer, Fehlermeldung "Gruppe existiert nicht"                                                                                           & \checkmark           &  &  &  \\
		Es gibt nur eine Gruppe und zu verschiebender Nutzer in dieser                                                                       & Dropdown neue  Gruppe leer, Fehlermeldung "Gruppe existiert nicht"                                                                                               & \checkmark           &  &  &  \\
		Subtab Gruppen erstellen und Kurs bearbeiten aufrufen                                                                                & Weiterführung auf die entsprechenden Sub tabs                                                                                                                    & \checkmark           &  &  &  \\
		
		
			\end{tabular}
\end{table}
\begin{table}[]
	\centering
	\caption{Testfälle}
	\label{table3}
	\begin{tabular}{p{6cm}p{7cm}p{3cm}llll}Verschiebe User der Gruppenadmin ist in eine andere Gruppe                                                                           & bleibt Gruppenadmin in Ausgangsgruppe und ist neues Gruppenmitglied in neuer Gruppe                                                                              & \checkmark          &  &  &  \\
		&                                                                                                                                                                  &             &  &  &  \\
		\textbf{	Tab Verlinkung  }                                                                                                                     &                                                                                                                                                                  &             &  &  &  \\
			&                                                                                                                                                                  &             &  &  &  \\
		Verlinke Übung in eine Gruppe                                                                                                        & Übung wurde in diese Gruppe verlinkt                                                                                                                             & \checkmark           &  &  &  \\
		Locator bzw Hierarchie im Verlinkungstab benutzen                                                                                    & Hierarchie führt einen an den beschrieben Ort                                                                                                                    & \checkmark           &  &  &  \\
		
		Verlinke Übung in alle 20 Gruppen obwohl in drei schon der Test verlinkt wurde                                                       & Verlinke in alle Gruppen in denen noch kein Test existiert und nur in den Admin Folder falls er existiert sonst einfach in die Gruppe                            & \checkmark           &  &  &  \\Verlinke Test in alle 20 Gruppen obwohl in drei schon der Test verlinkt wurde                                                        & Verlinke in alle Gruppen in denen noch kein Test existiert und nur in den Admin Folder falls er existiert sonst einfach in die Gruppe                            & \checkmark           &  &  &  \\
		In Kurs existieren keine Gruppen versuche Übung zu verlinken                                                                         & Zeige keine Gruppen in bestimmten Gruppe verlinken an und verlinkt auch nicht                                                                                    & \checkmark           &  &  &  \\
		&                                                                                                                                                                  &             &  &  &  \\
		Erzeuge Test                                                                                                                         & Verlinkung Tab wird sofort nach der Testerzeugung angezeigt                                                                                                      & \checkmark           &  &  &  \\
		Verlinke Test in 4 von 21 Gruppen aus Admin Folder obwohl in einer Gruppe kein Admin Folder existiert                               & Verlinke in alle ausgewählten Gruppen den Test                                                                                                                   & \checkmark           &  &  &  \\
		Verlinke Test in alle 21 Gruppen obwohl in vier schon der Test verlinkt wurde                                                        & Verlinke in alle Gruppen in denen noch kein Test existiert und nur in den Admin Folder falls er existiert sonst einfach in die Gruppe                            & \checkmark           &  &  &  \\
		Backtarget aus Verlinkungstab Test benutzen                                                                                          & Führt zurück auf Testinfo                                                                                                                                        & \checkmark           &  &  &  \\
		Locator bzw. Hierarchie im Verlinkstab Test benutzen                                                                                 & Führt zurück auf erwartetes Ziel                                                                                                                                 & \checkmark           &  &  &  \\
		In Kurs existieren keine Gruppen versuche Test zu verlinken                                                                          & Zeige keine Gruppen in bestimmten Gruppe verlinken an und verlinkt auch nicht                                                                                    & \checkmark           &  &  &  \\
		Verlinke Test in eine Gruppe                                                                                                         & Test wurde in diese Gruppe verlinkt                                                                                                                              & \checkmark           &  &  &  \\
	
	\end{tabular}
\end{table}


\begin{table}[]
	\centering
	\caption{Testfälle}
	\label{table4}
	\begin{tabular}{p{6cm}p{7cm}p{3cm}}
			\textbf{Gruppenfilter}  &   &   \\
		    Öffne Tab Gruppenfilter	& Eine leere Tabelle wird angezeigt, in der Toolbar können alle Gruppen in denen man Schreibrechte besitzt ausgewählt werden, alle Übungseinheiten der Übung können ausgewählt werden &\checkmark \\
			Wähle eine Gruppe und eine Übungseinheit aus & Es werden in der Tabelle nur die Benutzer der ausgewählten Gruppe angezeigt. Alle abgegebenen Dateien werden angezeigt, die eingestellten Noten, Bewertungen, Notizen und Kommentare werden in der Tabelle angezeigt &\checkmark \\
			Speichere Eingaben & Alle Eingaben werden in die Datenbank geschrieben, die Gesamtnote der Übung wird neu berechnet. Es wird wieder die leere Tabelle ohne Auswahl angezeigt &\checkmark \\
			Markiere Benutzer und wähle \textit{Als Mail versenden}& Weiterleitung zum Mail Fenster, in dem als Empfänger alle ausgewählten Benutzer eingetragen sind  &\checkmark \\
			Mail versenden & Die Mail wird an alle Empfänger verschickt, es wird wieder die leere Tabelle angezeigt. &\checkmark \\
			Einzelne Abgabe herunterladen & Nach dem Klick auf \textit{Abgaben herunterladen} wird die Abgabe heruntergeladen &\checkmark \\
			Klick auf \textit{Zurück} & Der Benutzer wird zurück auf die Info Seite der Übung geleitet &\checkmark \\
			Klick auf die übergeordneten Elemente die am oberen Rand der Seite angezeigt werden & Der Benutzer wird zu dem jeweiligen Objekt weitergeleitet &\checkmark \\
			Auswählen einer Gruppe ohne Abgaben & Die Tabelle bleibt leer  & \checkmark  \\
			Auswählen einer Teamabgabe 	& Teams werden angezeigt, Auswahlmöglichkeiten für Team erstellen und Team auflösen erscheinen  & \checkmark  \\
			Benutzer ist in mehreren Gruppen Mitglied & Benutzer wir bei beiden Gruppen angezeigt  & \checkmark  \\
			Benutzer hat nichts abgegeben &  Keine Abgabe erscheint in der Tabelle & \checkmark  \\
			Benutzer besitzt keine Schreibrechte & Tab \textit{Gruppenfilter} wird nicht angezeigt  & \checkmark  \\
			Benutzer besitzt Schreibrechte, ist aber kein Gruppenadministrator & Der Benutzer bekommt keine Gruppen als Auswahlmöglichkeit angezeigt, es wird eine leere Tabelle angezeigt  & \checkmark  \\
			
			

	\end{tabular}
\end{table}

