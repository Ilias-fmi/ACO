\section{ilACOLinkGUI}

\subsection*{Beschreibung}
Diese Klasse implementiert die Funktion \textit{Verlinkung} und beinhaltet die Möglichkeit, Exercise- oder Test-Objekte in einem Kurs in alle oder nur in ausgewählte Gruppen in einem Kurs zu verlinken.

\subsection*{Klassenvariablen}
\subparagraph{ctrl}
protected - type: ilCtrl - Steuerung Zugriffsrechte
\subparagraph{tpl}
protected - type: ilTemplate - Template für Darstellung und Formatierung
\subparagraph{pl}
protected - type: ilACOPlugin - Instanz des Plugins
\subparagraph{tabs}
protected - type: ilTabsGUI - Verwaltung von Tabs
\subparagraph{ilLocator}
protected - type: ilLocatorGUI - Darstellung in der Tree Hierarchie
\subparagraph{lng}
protected - type: ilLanguage - Einbindung der Language-File von ILIAS
\subparagraph{tree}
protected - type: ilTree - Verwaltung der Tree Hierarchie

\subsection*{Funktions-Liste}
\subparagraph{\nameref{constructLGUI}}
\subparagraph{\nameref{prepareOutputLGUI}}
\subparagraph{\nameref{setTitleAndIconLGUI}}
\subparagraph{\nameref{executeCommandLGUI}}
\subparagraph{\nameref{viewLGUI}}
\subparagraph{\nameref{initFormLGUI}}
\subparagraph{\nameref{getAdminFolderIdsLGUI}}
\subparagraph{\nameref{getFolderNameLGUI}}
\subparagraph{\nameref{getParentIdsLGUI}}
\subparagraph{\nameref{getGroupFolderIDLGUI}}
\subparagraph{\nameref{linkLGUI}}
\subparagraph{\nameref{getGroupsLGUI}}
\subparagraph{\nameref{isCourseLGUI}}
\subparagraph{\nameref{isAlreadyLinkedLGUI}}
\subparagraph{\nameref{checkAccessLGUI}}

\subsection*{Funktionen}

\subsubsection*{\textit{\_\_construct}}\label{constructLGUI}
\subparagraph{Beschreibung}
\begin{itemize}
	\item[] \noindent\fbox{\_\_construct()} 
	\item[] Konstruktion der Grundinstanz
\end{itemize}

\subsubsection*{\textit{prepareOutput}}\label{prepareOutputLGUI}
\subparagraph{Beschreibung}
\begin{itemize}
	\item[] \noindent\fbox{prepareOutput()} 
	\item[] Grundlegende Eigenschaften werden festgelegt: Tabs, Position in der Tree Hierarchie, BackTarget, Titel und Icon, sowie das verwendete Template
\end{itemize}

\subsubsection*{\textit{setTitleAndIcon}}\label{setTitleAndIconLGUI}
\subparagraph{Beschreibung}
\begin{itemize}
	\item[] \noindent\fbox{setTitleAndIcon()} 
	\item[] Titel und Symbol des Objekt werden festgelegt
\end{itemize}

\subsubsection*{\textit{executeCommand}}\label{executeCommandLGUI}
\subparagraph{Beschreibung}
\begin{itemize}
	\item[] \noindent\fbox{executeCommand()} 
	\item[] Regelung der Umsetzung von Befehlen mittels CommandButtons
\end{itemize}

\subsubsection*{\textit{view}}\label{viewLGUI}
\subparagraph{Beschreibung}
\begin{itemize}
	\item[] \noindent\fbox{view()} 
	\item[] Ausgabe der initialen Ansicht als HTML-Dokument
\end{itemize}

\subsubsection*{\textit{initForm}}\label{initFormLGUI}
\subparagraph{Beschreibung}
\begin{itemize}
	\item[] \noindent\fbox{initForm()} 
	\item[] Formular wird generiert und mit initialen Werten gefüllt
\end{itemize}
\subparagraph{Rückgabewerte}
\begin{itemize}
\item[] \textbf{form} - Formular mit Standardwerten, type: ilPropertyFormGUI
\end{itemize}


\subsubsection*{\textit{getAdminFolderIds}}\label{getAdminFolderIdsLGUI}
\subparagraph{Beschreibung}
\begin{itemize}
	\item[] \noindent\fbox{getAdminFolderIds()} 
	\item[] Gibt alle ID's der Ordner in den aktuell per Checkbox ausgewählten Gruppen zurück die den selben Namen tragen wie der dem Objekt übergeordnete Ordner.
\end{itemize}
\subparagraph{Rückgabewerte}
\begin{itemize}
	\item[] \textbf{ids} - IDs der Ordner in den in der GUI ausgewählten Gruppen mit dem selben Namen des über dem Objekt liegendem Ordners, type: Array
\end{itemize}

\subsubsection*{\textit{getFolderName}}\label{getFolderNameLGUI}
\subparagraph{Beschreibung}
\begin{itemize}
	\item[] \noindent\fbox{getFolderName()} 
	\item[] Ausgabe des Titels des übergeordneten Ordners
\end{itemize}
\subparagraph{Rückgabewerte}
\begin{itemize}
	\item[] \textbf{folder['title']} - Titel des Ordners, type: String 
	\item[] \textbf{-1} - Falls kein Ordner über dem Objekt existiert, type: Integer
\end{itemize}

\subsubsection*{\textit{getParentIds}}\label{getParentIdsLGUI}
\subparagraph{Beschreibung}
\begin{itemize}
	\item[] \noindent\fbox{getParentIds(\$id)} 
	\item[] Ausgabe der ID des dem Objekt übergeordneten Objekts
\end{itemize}
\subparagraph{Parameter-Liste}
\begin{itemize}
	\item[] \textbf{id} - ID eines Objekts, type: Integer 
\end{itemize}
\subparagraph{Rückgabewerte}
\begin{itemize}
	\item[] \textbf{ids} - IDs zu den übergeordneten Objekten eines Objekts, type: Array
\end{itemize}

\subsubsection*{\textit{getGroupFolderID}}\label{getGroupFolderIDLGUI}
\subparagraph{Beschreibung}
\begin{itemize}
	\item[] \noindent\fbox{getGroupFolderID()} 
	\item[] Gibt die ID des Ordners in der Gruppe mit einem bestimmten Namen zurück.
\end{itemize}
\subparagraph{Parameter-Liste}
\begin{itemize}
	\item[] \textbf{group\_id} - ID der Gruppe, type: Integer
	\item[] \textbf{folder\_name} - Titel des Ordners, type: Integer
\end{itemize}
\subparagraph{Rückgabewerte}
\begin{itemize}
	\item[] \textbf{ids} - IDs der Ordner in einer Gruppe, type: Array
	\item[] \textbf{-2} - Falls kein solcher Ordner existiert, type: Integer
\end{itemize}

\subsubsection*{\textit{link}}\label{linkLGUI}
\subparagraph{Beschreibung}
\begin{itemize}
	\item[] \noindent\fbox{link()} 
	\item[] Verlinkt das gewählte Objekt in die entsprechenden Gruppen, nach der Überprüfung, ob es dort bereits existiert oder ob es in einen Ordner in der Gruppe oder direkt in die Gruppe verlinkt werden muss.
\end{itemize}

\subsubsection*{\textit{getGroups}}\label{getGroupsLGUI}
\subparagraph{Beschreibung}
\begin{itemize}
	\item[] \noindent\fbox{getGroups()} 
	\item[] Funktion liefert alle Gruppen in einem Kurs
\end{itemize}
\subparagraph{Parameter-Liste}
\begin{itemize}
	\item[] \textbf{ref\_id} - ReferenzID eines Kurses, type: Integer
\end{itemize}
\subparagraph{Rückgabewerte}
\begin{itemize}
	\item[] \textbf{data} - Titel und ReferenzIDs der Gruppen in einem Kurs, type: Array
\end{itemize}

\subsubsection*{\textit{isCourse}}\label{isCourseLGUI}
\subparagraph{Beschreibung}
\begin{itemize}
	\item[] \noindent\fbox{isCourse()} 
	\item[] Überprüft, ob das Objekt zu einer ReferenzID ein Kurs ist
\end{itemize}
\subparagraph{Parameter-Liste}
\begin{itemize}
	\item[] \textbf{ref\_id} - ReferenzID eines Kurses, type: Integer
\end{itemize}
\subparagraph{Rückgabewerte}
\begin{itemize}
	\item[] \textbf{(true, false)} - Ist Objekt ein Kurs? (ja/nein), type: Boolean
\end{itemize}

\subsubsection*{\textit{isAlreadyLinked}}\label{isAlreadyLinkedLGUI}
\subparagraph{Beschreibung}
\begin{itemize}
	\item[] \noindent\fbox{isAlreadyLinked()} 
	\item[] Funktion überprüft, ob das Objekt in eine Gruppe schon verlinkt wurde
\end{itemize}
\subparagraph{Parameter-Liste}
\begin{itemize}
	\item[] \textbf{folder\_ref\_id} - ReferenzID einer Gruppe, type: Integer
	\item[] \textbf{obj\_id} - ID des Objekts, das verlinkt werden soll, type: Integer
\end{itemize}
\subparagraph{Rückgabewerte}
\begin{itemize}
	\item[] \textbf{(true, false)} - Bereits verlinkt? (ja/nein), type: Boolean
\end{itemize}

\subsubsection*{\textit{checkAccess}}\label{checkAccessLGUI}
\subparagraph{Beschreibung}
\begin{itemize}
	\item[] \noindent\fbox{checkAccess()} 
	\item[] Überprüft, ob man Lese- oder Schreibrechte auf das aktuelle Objekt hat.
\end{itemize}