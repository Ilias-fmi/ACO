\section{ilACOGroupDisplayGUI}

\subsection*{Beschreibung}
Diese Klasse implementiert die Funktion \textit{Kurs bearbeiten} und beinhaltet die Möglichkeit, die einzelnen Gruppen in einem Kurs zentral zu verwalten. 
Hierzu werden bestimmte Parameter der Gruppen in einer tabellarischen Übersicht dargestellt und können editiert werden.

\subsection*{Klassenvariablen}
\subparagraph{ctrl}
protected - type: ilCtrl
\subparagraph{tpl}
protected - type: ilTemplate
\subparagraph{pl}
protected - type: ilACOPlugin
\subparagraph{tabs}
protected - type: ilTabsGUI
\subparagraph{ilLocator}
protected - type: ilLocatorGUI
\subparagraph{lng}
protected - type: ilLanguage
\subparagraph{tree}
protected - type: ilTree
\subparagraph{form}
protected - type: ilPropertyFormGUI
\subparagraph{groupAdmins}
protected - type: array

\subsection*{Funktions-Liste}
\subparagraph{\nameref{constructGDGUI}}
\subparagraph{\nameref{prepareOutputGDGUI}}
\subparagraph{\nameref{setTitleAndIconGDGUI}}
\subparagraph{\nameref{executeCommandGDGUI}}
\subparagraph{\nameref{viewGDGUI}}
\subparagraph{\nameref{initFormGDGUI}}
\subparagraph{\nameref{saveGroupsGDGUI}}
\subparagraph{\nameref{getUserIdGDGUI}}
\subparagraph{\nameref{getAdminRoleIdGDGUI}}
\subparagraph{\nameref{getLoginbyIDSGDGUI}}
\subparagraph{\nameref{getCourseAdminIDsGDGUI}}
\subparagraph{\nameref{loadDateGDGUI}}
\subparagraph{\nameref{updateGroupGDGUI}}
\subparagraph{\nameref{getTableDataGDGUI}}
\subparagraph{\nameref{doUserAutoCompleteGDGUI}}
\subparagraph{\nameref{checkAccessGDGUI}}

\subsection*{\textit{Funktionen}}

\subsubsection*{\_\_construct}\label{constructGDGUI}
\subparagraph{Beschreibung}
\begin{itemize}
	\item[] \noindent\fbox{\_\_construct()}
	\item[] 
\end{itemize}
\subparagraph{Parameter-Liste}
\subparagraph{Rückgabewerte}

\subsubsection*{prepareOutput}\label{prepareOutputGDGUI}
\subparagraph{Beschreibung}
\begin{itemize}
	\item[] \noindent\fbox{prepareOutput()}
	\item[] 
\end{itemize}
\subparagraph{Parameter-Liste}
\subparagraph{Rückgabewerte}

\subsubsection*{setTitleAndIcon}\label{setTitleAndIconGDGUI}
\subparagraph{Beschreibung}
\begin{itemize}
	\item[] \noindent\fbox{setTitleAndIcon()}
	\item[] 
\end{itemize}
\subparagraph{Parameter-Liste}
\subparagraph{Rückgabewerte}

\subsubsection*{executeCommand}\label{executeCommandGDGUI}
\subparagraph{Beschreibung}
\begin{itemize}
	\item[] \noindent\fbox{executeCommand()}
	\item[] 
\end{itemize}
\subparagraph{Parameter-Liste}
\subparagraph{Rückgabewerte}

\subsubsection*{view}\label{viewGDGUI}
\subparagraph{Beschreibung}
\begin{itemize}
	\item[] \noindent\fbox{view()}
	\item[] 
\end{itemize}
\subparagraph{Parameter-Liste}
\subparagraph{Rückgabewerte}

\subsubsection*{initForm}\label{initFormGDGUI}
\subparagraph{Beschreibung}
\begin{itemize}
	\item[] \noindent\fbox{initForm()}
	\item[] 
\end{itemize}
\subparagraph{Parameter-Liste}
\subparagraph{Rückgabewerte}

\subsubsection*{saveGroups}\label{saveGroupsGDGUI}
\subparagraph{Beschreibung}
\begin{itemize}
	\item[]  \noindent\fbox{saveGroups()} 
	\item[] Die Änderungen in der tabellarischen Übersicht der Gruppen werden an die Funktion \nameref{updateGroupGDGUI} übergeben. Anschließend wird die tabellarische Übersicht aktualisiert, um die aktuellen Daten anzuzeigen.
\end{itemize}
\subparagraph{Parameter-Liste}
\subparagraph{Rückgabewerte}

\subsubsection*{getUserId}\label{getUserIdGDGUI}
\subparagraph{Beschreibung}
\begin{itemize}
	\item[]  \noindent\fbox{getUserId(\$user\_login)} 
	\item[] Benutzername als Eingabe liefert die dazugehörige UserID
\end{itemize}
\subparagraph{Parameter-Liste}
\begin{itemize}
	\item[] \textbf{user\_login} - Benutzername, type: String
\end{itemize}
\subparagraph{Rückgabewerte}
\begin{itemize}
	\item[] \textbf{data[0]} - UserID, type: Integer
\end{itemize}

\subsubsection*{getAdminRoleId}\label{getAdminRoleIdGDGUI}
\subparagraph{Beschreibung}
\begin{itemize}
	\item[]  \noindent\fbox{getAdminRoleId(\$obj\_id)} 
	\item[] ObjektID einer Gruppe als Eingabe liefert die ObjektID der Benutzerrolle des Gruppen-Administrators
\end{itemize}
\subparagraph{Parameter-Liste}
\begin{itemize}
	\item[] \textbf{obj\_id} - ID eines Objekts, type: Integer
\end{itemize}
\subparagraph{Rückgabewerte}
\begin{itemize}
	\item[] \textbf{role\_id[0]} - Objekt ID der Benutzerrolle Gruppen-Administrator, type: Integer
\end{itemize}

\subsubsection*{getLoginbyIDS}\label{getLoginbyIDSGDGUI}
\subparagraph{Beschreibung}
\begin{itemize}
	\item[]  \noindent\fbox{getLoginByIDS(\$ids)} 
	\item[] Auf Eingabe von UserIDs werden die dazugehörigen Benutzernamen ausgegeben.
\end{itemize}
\subparagraph{Parameter-Liste}
\begin{itemize}
	\item[] \textbf{ids} - UserIDs, type: Integer[]
\end{itemize}
\subparagraph{Rückgabewerte}
\begin{itemize}
	\item[] \textbf{logins} - Benutzernamen, type: String[]
\end{itemize}

\subsubsection*{getCourseAdminIDs}\label{getCourseAdminIDsGDGUI}
\subparagraph{Beschreibung}
\begin{itemize}
	\item[]  \noindent\fbox{getCourseAdminIDs(\$ref\_id)} 
	\item[] Auf Eingabe der ReferenzID eines Kurses wird die UserID des dazugehörigen Kurs-Administrators ausgegeben.
\end{itemize}
\subparagraph{Parameter-Liste}
\begin{itemize}
	\item[] \textbf{ref\_id} - ReferenzID eines Kurses, type: Integer
\end{itemize}
\subparagraph{Rückgabewerte}
\begin{itemize}
	\item[] \textbf{ids} - UserID des Kurs-Administrators, type: Integer
\end{itemize}

\subsubsection*{loadDate}\label{loadDateGDGUI}
\subparagraph{Beschreibung}
\begin{itemize}
	\item[] \noindent\fbox{loadDate(\$a\_field)}
	\item[] 
\end{itemize}
\subparagraph{Parameter-Liste}
\subparagraph{Rückgabewerte}

\subsubsection*{updateGroup}\label{updateGroupGDGUI}
\subparagraph{Beschreibung}
\begin{itemize}
	\item[]  \noindent\fbox{updateGroups(\$obj\_id, \$title, \$description, \$tutor, \$members, \$reg\_start, \$reg\_end, \$time\_reg)} 
	\item[] Die Funktion wird von \nameref{saveGroupsGDGUI} aufgerufen und schreibt die geänderten Werte in die Datenbank.
\end{itemize}
\subparagraph{Parameter-Liste}
\begin{itemize}
	\item[] \textbf{obj\_id} - ObjektID der Gruppe, type: Integer
	\item[] \textbf{title} - Titel der Gruppe, type: String
	\item[] \textbf{description} - Beschreibung der Gruppe, type: String
	\item[] \textbf{tutor} - Benutername des Gruppentutors, type: String
	\item[] \textbf{members} - Anzahl maximaler Mitglieder, type: Integer
	\item[] \textbf{reg\_start} - Zeitpunkt Anfang Beitritt, type: ilDateTime
	\item[] \textbf{reg\_end} - Zeitpunkt Ende Beitritt, type: ilDateTime
	\item[] \textbf{time\_reg} - Zeitlich begrenzter Beitritt, type: Boolean
\end{itemize}
\subparagraph{Rückgabewerte}

\subsubsection*{getTableData}\label{getTableDataGDGUI}
\subparagraph{Beschreibung}
\begin{itemize}
	\item[] \noindent\fbox{getTableData(\$ref\_id)}
	\item[] Liefert auf Eingabe der ReferenzID eines Kurses bestimmte Parameter der Gruppen in diesem Kurs.
\end{itemize}
\subparagraph{Parameter-Liste}
\begin{itemize}
	\item[] \textbf{ref\_id} - ReferenzID des Kurses, type: Integer
\end{itemize}
\subparagraph{Rückgabewerte}
\begin{itemize}
	\item[] \textbf{data} - Parameter aller Gruppen zu dem Kurs, type: Array
\end{itemize}

\subsubsection*{doUserAutoComplete}\label{doUserAutoCompleteGDGUI}
\subparagraph{Beschreibung}
\begin{itemize}
	\item[] \noindent\fbox{doUserAutoComplete()}
	\item[] 
\end{itemize}
\subparagraph{Parameter-Liste}
\subparagraph{Rückgabewerte}

\subsubsection*{checkAccess}\label{checkAccessGDGUI}
\subparagraph{Beschreibung}
\begin{itemize}
	\item[] \noindent\fbox{checkAccess()}
	\item[] 
\end{itemize}
\subparagraph{Parameter-Liste}
\subparagraph{Rückgabewerte}