\section{ilACOGroupGUI}

\subsection*{Beschreibung}
In dieser Klasse können über ein Formular mehrere Gruppen auf einmal in einem Kurs angelegt werden. Außerdem können optionale ein oder mehrere Ordner im Kurs und den jeweiligen Gruppen erstellt werden. Folgende Eigenschaften stehen für die Gruppen bereit: Gruppenname (Präfix im Titel, der mit aufsteigenden Ziffern ergänzt wird), Anzahl der zu erstellenden Gruppen, maximale Mitgliederzahl, Beitritt direkt oder mit Passwortschutz, optional: Zeitlich begrenzter Zutritt und anlegen eines oder mehrerer Ordner.


\subsection*{Klassenvariablen}
\subparagraph{CREATION\_SUCCEEDED}
protected - type: String - Ausgabe Erstellung erfolgt
\subparagraph{CREATION\_FAILED}
protected - type: String - Ausgabe fehlgeschlagene Erstellung
\subparagraph{ctrl}
protected - type: ilCtrl - Steuerung Zugriffsrechte
\subparagraph{tpl}
protected - type: ilTemplate - Template für Darstellung und Formatierung
\subparagraph{pl}
protected - type: ilACOPlugin - Instanz des Plugins
\subparagraph{tabs}
protected - type: ilTabsGUI - Verwaltung von Tabs
\subparagraph{ilLocator}
protected - type: ilLocatorGUI - Darstellung in der Tree Hierarchie
\subparagraph{lng}
protected - type: ilLanguage - Einbindung der Language-File von ILIAS
\subparagraph{tree}
protected - type: ilTree - Verwaltung der Tree Hierarchie
\subparagraph{courses}
protected - type: Array - Verwaltung von Kursen
\subparagraph{members}
protected - type: Integer - Max. Mitgliederzahl in einer Gruppe
\subparagraph{group\_count}
protected - type: Integer - Anzahl an Gruppen
\subparagraph{reg\_proc}
protected - type: ilRadioGroupInputGUI - Auswahl Anmeldeart
\subparagraph{pass}
protected - type: String - Passwort bei Anmeldebeschränkung
\subparagraph{group\_name}
protected - type: String - Präfix der Gruppentitel
\subparagraph{group\_folder\_name}
protected - type: String - Titel des/der Ordners/Ordner 
\subparagraph{group\_folder\_name\_checkbox}
protected - type: ilCheckboxInputGUI - \\~ \hspace*{6,2cm} {Auswahl zur Erstellung von Ordner(n)}


\subsection*{Funktions-Liste}
\subparagraph{\nameref{constructGGUI}}
\subparagraph{\nameref{prepareOutputGGUI}}
\subparagraph{\nameref{setTitleAndIconGGUI}}
\subparagraph{\nameref{executeCommandGGUI}}
\subparagraph{\nameref{viewGGUI}}
\subparagraph{\nameref{loadDateGGUI}}
\subparagraph{\nameref{countGroupsGGUI}}
\subparagraph{\nameref{createGroupsGGUI}}
\subparagraph{\nameref{groupsInCourseGGUI}}
\subparagraph{\nameref{folderAlreadyExistingCourseGGUI}}
\subparagraph{\nameref{folderAlreadyExistingGroupGGUI}}
\subparagraph{\nameref{checkAccessGGUI}}


\subsection*{Funktionen}

\subsubsection*{\textit{\_\_construct}}\label{constructGGUI}
\subparagraph{Beschreibung}
\begin{itemize}
	\item[] \noindent\fbox{\_\_construct()}
	\item[] Konstruktion der Grundinstanz
\end{itemize}

\subsubsection*{\textit{prepareOutput}}\label{prepareOutputGGUI}
\subparagraph{Beschreibung}
\begin{itemize}
	\item[] \noindent\fbox{prepareOutput()}
	\item[] Grundlegende Eigenschaften werden festgelegt: Tabs, Position in der Tree Hierarchie, BackTarget, Titel und Icon, sowie das verwendete Template
\end{itemize}

\subsubsection*{\textit{setTitleAndIcon}}\label{setTitleAndIconGGUI}
\subparagraph{Beschreibung}
\begin{itemize}
	\item[] \noindent\fbox{setTitleAndIcon()}
	\item[] Titel und Symbol des Objekt werden festgelegt
\end{itemize}

\subsubsection*{\textit{executeCommand}}\label{executeCommandGGUI}
\subparagraph{Beschreibung}
\begin{itemize}
	\item[] \noindent\fbox{executeCommand()}
	\item[] Regelung der Umsetzung von Befehlen mittels CommandButtons
\end{itemize}

\subsubsection*{\textit{view}}\label{viewGGUI}
\subparagraph{Beschreibung}
\begin{itemize}
	\item[] \noindent\fbox{view()}
	\item[] Ausgabe der initialen Ansicht als HTML-Dokument
\end{itemize}

\subsubsection*{\textit{loadDate}}\label{loadDateGGUI}
\subparagraph{Beschreibung}
\begin{itemize}
	\item[] \noindent\fbox{loadDate(\$a\_field)}
	\item[] Datum im Formular wird angepasst, um es in die Datenbank schreiben zu können.
\end{itemize}
\subparagraph{Parameter-Liste}
\begin{itemize}
	\item[] \textbf{a\_field} - Wert eines Datumsfelds im Formular, type: ilDateTime
\end{itemize}
\subparagraph{Rückgabewerte}
\begin{itemize}
	\item[] \textbf{date} - Wert aus dem Feld umgewandelt in Datenbank-kompatibles Format, type: ilDateTime
\end{itemize}

\subsubsection*{\textit{countGroups}}\label{countGroupsGGUI}
\subparagraph{Beschreibung}
\begin{itemize}
	\item[] \noindent\fbox{countGroups(\$parent\_id)}
	\item[] Gibt die Anzahl an Gruppen zurück, die in einem Kurs vorhanden sind. Anmerkung: Wir müssen in die tree table schauen weil crs item nicht sofort aktualisiert wird.
\end{itemize}
\subparagraph{Parameter-Liste}
\begin{itemize}
	\item[] \textbf{parent\_id} - ReferenzID des Kurses, type: Integer
\end{itemize}
\subparagraph{Rückgabewerte}
\begin{itemize}
	\item[] \textbf{result} - Anzahl an Gruppen im Kurs, type: Integer
\end{itemize}

\subsubsection*{\textit{createGroups}}\label{createGroupsGGUI}
\subparagraph{Beschreibung}
\begin{itemize}
	\item[] \noindent\fbox{createGroups()}
	\item[] Funktion erstellt die Gruppen und eventuell Ordner in einem Kurs entsprechend den Angaben im Formular. 
\end{itemize}

\subsubsection*{\textit{groupsInCourse}}\label{groupsInCourseGGUI}
\subparagraph{Beschreibung}
\begin{itemize}
	\item[] \noindent\fbox{groupsInCourse()}
	\item[] Gibt die ReferenzIDs der Gruppen in einem Kurs zurück.
\end{itemize}
\subparagraph{Rückgabewerte}
\begin{itemize}
	\item[] \textbf{group\_id} - ReferenzIDs der Gruppen im Kurs, type: Array
\end{itemize}

\subsubsection*{\textit{folderAlreadyExistingCourse}}\label{folderAlreadyExistingCourseGGUI}
\subparagraph{Beschreibung}
\begin{itemize}
	\item[] \noindent\fbox{folderAlreadyExistingCourse(\$folder\_name)}
	\item[] Überprüft, ob ein Ordner im Kurs bereits vorhanden ist.
\end{itemize}
\subparagraph{Parameter-Liste}
\begin{itemize}
	\item[] \textbf{folder\_name} - Titel des Ordners, type: String
\end{itemize}
\subparagraph{Rückgabewerte}
\begin{itemize}
	\item[] \textbf{(true,false)} - Ordner vorhanden? (ja/nein), type: Boolean
\end{itemize}

\subsubsection*{\textit{folderAlreadyExistingGroup}}\label{folderAlreadyExistingGroupGGUI}
\subparagraph{Beschreibung}
\begin{itemize}
	\item[] \noindent\fbox{folderAlreadyExistingGroup(\$folder\_name,\$group\_ref\_id)}
	\item[] Überprüft, ob ein Ordner in einer Gruppe bereits vorhanden ist.
\end{itemize}
\subparagraph{Parameter-Liste}
\begin{itemize}
	\item[] \textbf{folder\_name} - Titel des Ordners, type: String 
	\item[] \textbf{ref\_id} - ReferenzID der Gruppe, type: Integer
\end{itemize}
\subparagraph{Rückgabewerte}
\begin{itemize}
	\item[] \textbf{(true,false)} - Ordner vorhanden? (ja/nein), type: Boolean
\end{itemize}

\subsubsection*{\textit{checkAccess}}\label{checkAccessGGUI}
\subparagraph{Beschreibung}
\begin{itemize}
	\item[] \noindent\fbox{checkAccess()}
	\item[] Überprüft, ob man Lese- oder Schreibrechte auf das aktuelle Objekt hat.
\end{itemize}